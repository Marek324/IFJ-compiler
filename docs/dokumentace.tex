\documentclass[12pt]{article}
 
\usepackage[margin=1in]{geometry} 
\usepackage{amsmath,amsthm,amssymb}
\usepackage{graphicx}
\usepackage{attachfile}

\graphicspath{ {images/} }
 

\begin{document}
 
% --------------------------------------------------------------
%                         Start here
% --------------------------------------------------------------
 
\title{\Huge \textbf{Tým xhricma00 varianta vv-BVS}}

\author{\bf Marek Hric \\
\bf xhricma00
\and Mikuláš Lešiga\\
 xlesigm00
\and Roman Andraščík \\
 xandrar00
\and Adam Veselý \\ 
xvesela00
}

\maketitle
\bigskip
\begin{center}
 \Large \textbf{Rozdelenie bodov}  \normalsize
 \\ 
\medskip
xhricma00: 25\% \\
xlesigm00: 25\% \\
xandrar00: 25\% \\
xvesela00: 25\%  \\
\bigskip
 \Large \textbf{Rozšírenia} \normalsize
\\
\medskip
ORELSE\\
UNREACHABLE\\
BOOLTHEN\\
FOR\\
WHILE\\
FUNEXP\\
\end{center}

\newpage

\noindent \Large \textbf{Rozdelenie prace :} \\
\noindent\makebox[\linewidth]{\rule{\textwidth}{0.4pt}}
\newline \\
\indent \large Marek Hric : \normalsize
\paragraph{Lorem ipsum dolor sit amet, consectetur adipiscing elit, sed do eiusmod tempor incididunt ut labore et dolore magna aliqua. Ut enim ad minim veniam, quis nostrud exercitation ullamco laboris nisi ut aliquip ex ea commodo consequat. Duis aute irure dolor in reprehenderit in voluptate velit esse cillum dolore eu fugiat nulla pariatur. Excepteur sint occaecat cupidatat non proident, sunt in culpa qui officia deserunt mollit anim id est laborum \newline \\} 
\large Mikuláš Lešiga :
\normalsize
\paragraph{Lorem ipsum dolor sit amet, consectetur adipiscing elit, sed do eiusmod tempor incididunt ut labore et dolore magna aliqua. Ut enim ad minim veniam, quis nostrud exercitation ullamco laboris nisi ut aliquip ex ea commodo consequat. Duis aute irure dolor in reprehenderit in voluptate velit esse cillum dolore eu fugiat nulla pariatur. Excepteur sint occaecat cupidatat non proident, sunt in culpa qui officia deserunt mollit anim id est laborum \newline \\}
\large Roman Andraščík :
\normalsize
\paragraph{Lorem ipsum dolor sit amet, consectetur adipiscing elit, sed do eiusmod tempor incididunt ut labore et dolore magna aliqua. Ut enim ad minim veniam, quis nostrud exercitation ullamco laboris nisi ut aliquip ex ea commodo consequat. Duis aute irure dolor in reprehenderit in voluptate velit esse cillum dolore eu fugiat nulla pariatur. Excepteur sint occaecat cupidatat non proident, sunt in culpa qui officia deserunt mollit anim id est laborum \newline \\}
\large Adam Veselý :
\normalsize
\paragraph{Lorem ipsum dolor sit amet, consectetur adipiscing elit, sed do eiusmod tempor incididunt ut labore et dolore magna aliqua. Ut enim ad minim veniam, quis nostrud exercitation ullamco laboris nisi ut aliquip ex ea commodo consequat. Duis aute irure dolor in reprehenderit in voluptate velit esse cillum dolore eu fugiat nulla pariatur. Excepteur sint occaecat cupidatat non proident, sunt in culpa qui officia deserunt mollit anim id est laborum \newline \\}
 
\newpage

\noindent \Large \textbf{Diagram konečného automatu :}
\newline \\

\includegraphics[width=0.8\textwidth,scale=0.5]{fsm}

\newpage

 \Large \textbf{LL-gramatika :} \\ \normalsize
\noindent\makebox[\linewidth]{\rule{\textwidth}{0.4pt}}
\begin{enumerate}
\item aaaa
\item bbbb
\item ccccc
\item ddddddddd
\end{enumerate}


 \Large \textbf{LL-tabulka :}
\newline \\

\includegraphics[width=0.3\textwidth,scale=0.3]{LLtabulka}

 \Large \textbf{Precendecna-tabulka :}
\newline \\

\includegraphics[width=0.3\textwidth,scale=0.3]{Ptabulka}


\newpage

 \Large \textbf{Lexikalna analyza} \normalsize \\
\noindent\makebox[\linewidth]{\rule{\textwidth}{0.4pt}}

\paragraph{\indent Riesenie lexikalnej analyzy sme zacali vytvorenim diagramu deterministickeho konecneho automatu. Nasledne sme na jeho zaklade zacali vypracovavat implementaciu. Implementacia sa nachadza v subore \textit{scanner.c,} ktory pracuje s tokenmi deklarovanymi v subore \textit{token.h}. Hlavnou funkciou \textit{scanner.c} je funkcia \textit{get\_token}. Pre ulahcenie prace a prehladnosti kodu sme si deklarovali niekolko makier, ktore su extensivne pouzivane v hlavnej funkcii. Funkcia \textit{get\_token} berie postupne znaky z standardneho vstupu a vytvara token. Tokenu je priradeny jeho typ a hodnota ktora mu odpoveda. Funkcia zacina urcovanim jednoznakovych tokenov, ktore vie urcite hned na zaciatku. Pokracuje identifikaciou komentarov, ktore nasledne ignoruje. Po identifikacii komentarov zistuje ci sa jedna o ID alebo Keyword, pri klucovych slovach sa nasledne urcuje aj ich typ. Ak sa nejedna ani o jedno pokracuje kontrolov datovych typov pri ktorych uklada aj ich hodnoty.  \newline \\}

 \Large \textbf{Syntakticka analyza}\normalsize \\
\noindent\makebox[\linewidth]{\rule{\textwidth}{0.4pt}}

\paragraph{Riesenie syntaktickej analyzy sme zapocali vytvorenim LL gramatiky, LL tabulky a precendecnej tabulky. Nasledne na ich zaklade sme vypracovali subor \textit{parser.c a exp\_parser.c}.Tieto subory pracuju s uzlamy deklarovanymi v subore \textit{ast.h}. Spustenie syntaktickej analyzy zapocne zavolanim funkcie \textit{Parse()}. Tato funkcia postupne prechadza cez tokeny a priradzuje ich do uzlov pomocou ktorych postupne tvori abstraktny syntakticky strom na zaklade LL gramatiky. Subor \textit{parser.c} dalej riadi aj precedencnu analyzu volanim funkcii zo suboru \textit{exp\_parser.c}.Tento subor vytvori strom vyrazov, ktory je nasledne pripojeny do syntaktickeho stromu.  \newline \\}

 \Large \textbf{Semanticka analyza}\normalsize \\
\noindent\makebox[\linewidth]{\rule{\textwidth}{0.4pt}}

\paragraph{Semanticka analyza je implementovana v suboroch \textit{sem\_anal.c , symtable.c, sem\_anal.h a symtable.h}. Spustenie semantickej analyzy zapocne zavolanim funkcie \textit{analyse()}.Tato funkcia prechadza vytvoreny AST a postupne urcuje ci vyhovuje pravidlam jazyka IFJ24.Vyhovujuce funkcie su nasledne vlozene do tabulky symbolov, ktora je deklarovana v subore \textit{symtable.h}. \newline \\}

 \Large \textbf{Generovanie kodu} \normalsize \\
\noindent\makebox[\linewidth]{\rule{\textwidth}{0.4pt}}

\paragraph{Generator je implementovany v suboroch \textit{codegen\_priv.h. codegen.h a codegen.c }.Spustenie generacie kodu zapocne zavolanim funkcie \textit{codegen()}. \newline \\}

 \Large \textbf{Datove struktury}\normalsize \\
\noindent\makebox[\linewidth]{\rule{\textwidth}{0.4pt}}

\paragraph{\large \underline{Circular Buffer} \\ Implementovane v suboroch \textit{circ\_buff.c circ\_buff.h}. \\ \newline
Implementacia Circular Buffer je vyuzita hlavne v casti Scanner kde sluzi na bezpreblemove ziskavanie dat a ich naslednu validaciu. Na pracu so scannerom ho neskor vyuzivaju aj casti Parser a Expression Parser. Struktura obsahuje klasicke funkcie \textit{circ\_buff\_init, circ\_buff\_free, circ\_buff\_enqueue, circ\_buff\_dequeue, circ\_buff\_is\_empty}. 
\newline \\}

\paragraph{\large \underline{Dynamic String} \\ Implementovane v suboroch \textit{dyn\_str.c, dyn\_str.h}. \\ \newline
Implementacia dynamickeho retazca je vyuzita hlavne v Scanner casti programu kde sprostredkuvava validaciu a uschovavanie dat, neskor je pouzita aj v casti Codegen kde sluzi na ulahcenie validacie dat. Struktura dynamickeho retazca obsahuje klasicke funkcie \textit{dyn\_str\_init, dyn\_str\_grow, dyn\_str\_append, dyn\_str\_append\_str a dyn\_str\_free}.  
\newline \\}

\paragraph{\large \underline{Stack} \\ Implementovane v suboroch \textit{stack.c, stack.h}. \\ \newline
Implementaciu nasho zasobniku vyuzivame  v Expression Parser casti programu. Struktura zasobniku je implementovana s klasickymi funkciami \textit{stackInit, stackPush, stackPop, stackIsEmpty, stackClear a stackGetTop}. Zasobnik sme zvolili pre jeho optimalny pristup k datam a zachovanie jednoduchosti kodu.
\newline \\}

\end{document}
